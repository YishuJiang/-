\documentclass[lang=cn,10pt]{elegantbook}
\usepackage{ctex}
\title{中级微观经济学(拔尖班)笔记}
\subtitle{2024-2025学年秋冬学期}

\author{Yishu Jiang}
\institute{School of Economics,Zhejiang University}
\date{\today}

\extrainfo{Talk is cheap,show me your solution!}

\setcounter{tocdepth}{3}

\logo{ZJU.png}
\cover{Cover.png}

% 本文档命令
\usepackage{array}
\newcommand{\ccr}[1]{\makecell{{\color{#1}\rule{1cm}{1cm}}}}

% 修改标题页的橙色带
% \definecolor{customcolor}{RGB}{32,178,170}
% \colorlet{coverlinecolor}{customcolor}

\setcounter{tocdepth}{2}
\begin{document}

\maketitle
\frontmatter
\chapter*{前言}
\markboth{Introduction}{前言}
{\fangsong 
    本笔记是2024-2025学年秋冬学期面向经济学拔尖班开设的中级微观经济学的学习笔记,主要使用的教材为范里安的《微观经济学:现代观点》,授课老师为汪淼军.

    由于本人初次学习中微且水平能力有限,笔记中有所疏漏处,恳请指正.
}
\newpage

\tableofcontents

\mainmatter
\chapter{消费者行为}
基本知识框架:
\begin{enumerate}
    \item 偏好关系与效用函数
    \item 消费集、预算集
    \item 约束下的最优化问题;无差异曲线、Walras需求函数、效用最大化问题、支出最小化问题、间接效用函数、花费函数、对偶关系.
\end{enumerate}
\newpage
\section{消费集和预算约束}
\subsection{消费集}
首先我们讨论消费者选择的对象——商品.不考虑任何经济上的约束,对于可供消费者选择的商品,假设共有$n$种.用一个$n$维向量$x=(x_1,\cdots,x_n)$表示一个消费选择,称作\textbf{消费束},所有消费束所构成的集合$X\subset \mathbb{R}^n_{+}$称作\textbf{消费集}.

消费集通常符合以下假设:
\begin{enumerate}
    \item $X\subset \mathbb{R}_{+}^{n}$,即消费数量非负(“减少一些东西”可以用绝对值替代).
    \item $0\in X$.
    \item $X$是\textbf{闭集},指$\partial X\subset X$.即消费集允许"极限行为"“擦边球行为”.
    \item $X$是\textbf{凸集},指$\forall x^1,x^2\in X,\forall \lambda\in [0,1],\lambda x^1+(1-\lambda)x^2\in X$,换言之:\textbf{消费集中任意两点连线上的所有点都在消费集内部}.即可以通过连续的调整,实现一种消费向另一种消费的过渡.
    \item $X$无上界.即消费集仅仅表示消费者客观上可选择的商品组合,不考虑消费是否能够实现.
\end{enumerate}
\subsection{预算约束}
    进一步地,我们引入经济约束——商品的价格和人们的收入限制了消费.设价格向量$p=(p_1,\cdots,p_n)$;设消费这的预算为$m$,则有$p\cdot x\leq m$.在这一限制条件下,所有可行的消费品集合为$\{x\in X\mid p\cdot x\leq m\}$,它是消费集的子集——消费者能负担得起的所有消费束的集合,也就是\textbf{预算集}.

    当然,预算集能表示成$\{x\in X\mid p\cdot x\leq m\}$,得建立在\textbf{市场完备性}(所有商品的价格都是公开、透明的)的基础上.考虑最简单的情况,价格$p$不变,这种最简单的预算集被称为Walrasian运算集,这建立在\textbf{价格接受}假设上,仅当单个消费者的需求占总需求的占比很小时才成立.

    此外,还有一些因素也会影响消费者选择的可行域:比如资源约束、分配方式(配额等).

    预算集的边界$\{x\in X\mid p\cdot x=m\}$是$n$为空间中的$n-1$为超平面,称为\textbf{预算超平面}.可以看出,价格向量$p=(p_1,\cdots,p_n)$与预算超平面正交.
\newpage
\section{偏好关系和效用函数}
\begin{definition}
偏好关系$\succ$是消费集$X$上的一个二元关系,$\forall x,y\in X,x\succeq y$当且仅当“$X$至少和$y$一样好”.
\end{definition}
由此引申出另两种二元关系:
\begin{definition}[严格偏好关系]
    $x\succ y$当且仅当$x\succeq y$但$y\succeq x$不成立.
\end{definition}
\begin{definition}[无差异关系]
    $x\sim y$当且仅当$x\succeq y$且$y\succeq x$.
\end{definition}
\newpage
\section{消费者的最优选择}
在给定的约束下,理性的消费者会选择自己最喜欢的商品组合.这样的最优化结果有两种刻画指标:给定支付能力(收入),获取最大效用——效用最大化问题(Utility Maximizing Problem,UMP);给定效用,使用最小的支出——支出最小化问题(Expenditure Minimizing Problem,EMP).这两种问题在最优化问题的意义上且具有\textbf{对偶性}.

\subsection{效用最大化问题(UMP)}
\newpage

\subsection{支出最小化问题(EMP)}
\newpage

\subsection{对偶性: UMP与EMP的联系}
\newpage

\section{基于最优选择的进一步分析}
\newpage

\subsection{收入效应与替代效应}
\newpage

\subsection{福利分析}
\newpage

\subsection{加总和需求}
\newpage

\section{其他问题}
\newpage

\subsection{不确定性下的选择}
\newpage

\subsection{跨期选择}
这一部分内容在中级宏观经济学中也有讲到.
\newpage

\chapter{生产者行为}
\chapter{一般均衡理论}
\chapter{博弈论基础}
\chapter{市场结构分析}
\chapter{市场失灵}
\end{document}
