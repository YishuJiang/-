\documentclass[lang=cn,10pt]{elegantbook}
\usepackage{ctex}
\title{中级宏观经济学(拔尖班)笔记}
\subtitle{2024-2025学年秋冬学期}

\author{Yishu Jiang}
\institute{School of Economics,Zhejiang University}
\date{\today}

\extrainfo{Talk is cheap,show me your solution!}

\setcounter{tocdepth}{3}

\logo{ZJU.png}
\cover{Cover.png}

% 本文档命令
\usepackage{array}
\newcommand{\ccr}[1]{\makecell{{\color{#1}\rule{1cm}{1cm}}}}

% 修改标题页的橙色带
% \definecolor{customcolor}{RGB}{32,178,170}
% \colorlet{coverlinecolor}{customcolor}

\setcounter{tocdepth}{2}
\begin{document}

\maketitle
\frontmatter
\chapter*{前言}
\markboth{Introduction}{前言}
{\fangsong 
    本笔记是2024-2025学年秋冬学期面向经济学拔尖班开设的中级宏观经济学的学习笔记,主要使用的教材为何樟勇《高级宏观经济学》和卢卡斯的《经济动态的递归方法》,授课老师为邬介然.

    由于本人初次学习中宏且水平能力有限,笔记中有所疏漏处,恳请指正.
}
\newpage

\tableofcontents

\mainmatter
\part{中宏的简单模型}
\chapter{单期代表性行为人模型}
本章介绍最简单的一类宏观经济模型Representative Agent Model,在这个模型中,抽象掉\textbf{异质性和分配问题}(例如收入、财富、偏好等都被同质化假设掉了).
假定经济由一个代表性的企业和一个代表性的消费者组成(这和有许多本质相同的企业和许多本质相同的消费者所组成的经济是相同的).

基本的分析思路是:首先关注行为人在约束条件下的最优化问题,然后分析不同行为人的自利行为交互会产生怎样的效果.首先我们需要知道消费者的偏好,企业的生产技术以及消费者与企业的资源禀赋.

在正式开始分析前, 我们先讨论一个非常基本的问题: 为什么我们可以用代表性行为人的假设抽象掉所有的异质性?事实上, 它的理论基础是Arrow-Debru定理.当然,这里不深究具体的技术细节和定理证明,主要探讨思路上的问题:
在宏观的视角下,先不考虑个体偏好的差异,考虑更加易于度量的“收入和财富”的异质性,并试图抽象掉它——引入所谓的“个体冲击”,这里我们不管中间过程如何,把生下来到长大这段时间中的乱七八糟的都忽略掉.
\section{决策环境:偏好、禀赋与技术}
\subsection{基本环境}
\begin{enumerate}
    \item 经济活动只进行\textbf{一期}
    \item 经济中只有一个代表性消费者和一个代表性企业
    \item 代表性消费者决定最优的资本供给和劳动供给;代表性企业决定最优的资本和劳动使用量
    \item 消费者和企业的行为都是\textbf{竞争性}的(价格外生给定)
    \item 消费者拥有企业
\end{enumerate}
\subsection{偏好}
我们用效用函数来表示消费者的偏好:
$$u(c,l)$$
其中$c$表示消费,$l$是闲暇. 两者之间存在一个替代作用:每天的时间是固定的,闲暇时间越长,劳动时间越短,收入就越少,消费也就越少,反之同理.
效用函数满足以下假设:
\begin{enumerate}
   \item 严格凹、二次可微——边际效用递减;
   \item 严格增——$u(\cdot)$对$c$和$l$增;
   \item Inada条件:$\frac{\partial u}{\partial (\cdot)}(0,\cdot)=+\infty,\frac{\partial u}{\partial }$
\end{enumerate}
\newpage

\chapter{两期动态模型}
\newpage

\chapter{三期动态模型}
\newpage

\chapter{无穷期动态模型}
\newpage

\part{数学工具}
\chapter{动态规划Dynamic Programing}

\chapter{基本的数值计算方法}

\part{DEGE:动态随机一般均衡}
\chapter{真实经济周期模型RBC}

\chapter{Banking Model}

\chapter{金融摩擦}
\end{document}
